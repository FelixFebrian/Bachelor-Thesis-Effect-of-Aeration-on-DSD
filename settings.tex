% ============= Document class & settings =============
\documentclass[
	a4paper, 
	11pt, 
	headlines=1.25, 
	footlines=1.25, 
	DIV = 12, 
	numbers = noenddot,
	]{scrreprt}
\linespread{1}                			% Zeilenabstand

\usepackage{geometry}%[margin=2.5cm]{geometry}	% Seitenränder über das geometry-Paket einstellen
\geometry{										% unten 2 cm, sonst 2,5 cm
    top=2cm,
    bottom=2cm,
    left=2cm,
    right=2cm,
    		%For printing outer=2cm, inner=3cm setzen
    		%outer=2cm,
    		%inner=3cm,
}
%	BCOR = 0mm, 
%	titlepage=firstiscover, 
%	listof=totoc,  
%	bibliography=totoc					% Lines were in document class before

% Dokumentinformationen
\usepackage[
	pdftitle={Bachelor's Thesis},
	pdfsubject={Analysis of Droplets Size Distribution in an Aerated Stirred Tank Reactor},
	pdfauthor={Felix Febrian},
	pdfkeywords={},	
	hidelinks %Links nicht einrahmen
]{hyperref}

% ============= Packages =============
\usepackage[utf8]{inputenc}
%\usepackage[T1]{fontenc}
%\usepackage{lmodern}
\usepackage[english]{babel}		    % Umlaute, usw.
\usepackage{microtype}          			% besserer Schriftsatz
\usepackage{csquotes}           			% für Wörter in "[Text]": \enquote{}
\usepackage{amssymb}            			% ermöglicht mathematische Symbole
\usepackage{amsthm}             			% ermöglicht mathematische Umgebungen
\usepackage{amsmath}            			% ""
%\usepackage{mathpazo}           			% Schriftart für Mathe
\usepackage{amsfonts}
\usepackage{graphicx}           			% ermöglicht Einbinden von Grafiken
	\graphicspath{{pictures/}}  			% sagt graphicx wo die Grafiken liegen
\usepackage{tikz} 	        				% Zum Bild Ausrichten
%\usepackage{pxfonts}            			% für griechische Buchstaben
\usepackage{scrdate}
\usepackage{booktabs}           			% Für Tabellen-Linien
\usepackage{xcolor}             			% Farbiger Text
\usepackage{multirow}           			% bessere Tabellen
\usepackage{pdflscape}          			% um Tabellen um 90 Grad zu drehen
\usepackage{adjustbox}          			% Um Tabellen auf eine Seite zu beschränken
\usepackage{todonotes}					    % Enables inserting notes on the Document
\usepackage{xspace}						    % Paket für ein intelligenter Leerzeichen
\usepackage{tabto}						    % Enable using tab in the line
\usepackage[version=4]{mhchem}              % Enable writing chemical formula
\usepackage{chemformula}                    % --"--
\usepackage{chemfig}                        % To draw chemical molecules, especially organic molecules
\usepackage{pdfpages}                       % Enables inserting pdf
\usepackage{xfrac}                          % other fraction types
\usepackage[Bjornstrup]{fncychap}           % Fancy Chapter Styles, Options: Sonny, Lenny, Glenn, Conny, Rejne, Bjarne, Bjornstrup
\usepackage{nameref}

% ============= BibTex Options =============
\usepackage[authoryear, round, sort]{natbib}		% Package for bibtex -- citation Test, et.al (2000)
\usepackage{babelbib}					            % Package für die deutsche BibTeX-Umgebung ... (sonst kann man keine Auflage eines Buches angeben)
\usepackage{tocbibind}					            % List Bibliography in the Table of Contents

% ============= Packages with renewed command (MACRO) ================
\usepackage[intoc]{nomencl}
\makenomenclature
\usepackage{etoolbox}
    \renewcommand\nomgroup[1]{%
        \item[\bfseries
        \ifstrequal{#1}{L}{Latin Symbols}{%
        \ifstrequal{#1}{G}{Greek Symbols}{%
        \ifstrequal{#1}{N}{Non-Dimensional Numbers}{%
        \ifstrequal{#1}{A}{Abbreviations}{}}}}%
        ]}
    \newcommand{\nomunit}[1]{%
        \renewcommand{\nomentryend}{\dotfill{} #1}}


\usepackage{enumitem}
	\newcounter{inlineenum}                         % enumeration inside the text.
	\renewcommand{\theinlineenum}{\alph{inlineenum}}
	\newenvironment{inlineenum}
  		{\unskip\ignorespaces\setcounter{inlineenum}{0}%
   		\renewcommand{\item}{\refstepcounter{inlineenum}{\textit{\theinlineenum})~}}}
  		{\ignorespacesafterend}
  		
\usepackage{pgfplots}
  	\pgfplotsset{compat=newest}
  	\usetikzlibrary{plotmarks}
  	\usetikzlibrary{arrows.meta}
  	\usepgfplotslibrary{patchplots}
  	\usepackage{grffile}
	\pgfplotsset{plot coordinates/math parser=false}
\newlength\figH
\newlength\figW
\setlength{\figH}{10cm} % Hier kann die Größe der Plots gesetzt werden. default: 7 
\setlength{\figW}{13.33333cm} % Breite = 4/3*Höhe. default: 9.33333

% MAKRO pgfplot picture, #1 : .tex data, #2 : caption, #3 : Label
\newcommand{\plot}[3]{
    \begin{figure}[htbp]
        \centering
        {\input{#1}}
        %\vspace{0.1cm}
        \caption{#2}
        \label{#3}
    \end{figure}}
    
% Inserting pdf as figure
\newcommand{\vectorGraphic}[3]{
    \begin{figure}[htbp]
        \centering
        \includegraphics[scale=1]{#1}
        \caption{#2}
        \label{#3}
    \end{figure}}
    
%\renewcommand\thechapter{Chapter \arabic{chapter}}   