================= Introduction ======================

Mixing and stirring is one of the important and integral unit operation in the chemical and process engineering
Human beings have been cultivating microbiologies for centuries, adding values to consumer products from foods and beverages to pharmaceutical substances. As early as 9000 years ago, the people in pre-historic China had used mixed microbial culture to make wine from rice, honey and fruits. Due to developments in the biotechnology, biomanufacturing products advanced and became more complex. Production of protein drugs and other metabolites by means of cell cultures and DNA-Recombinant technology was achieved on an industrial scale. \cite{Zhang2017} tried to classify key findings in bioengineering and presented recent research trends in the field. The so called Biomanufacturing 4.0 aims to invent new products such as regenerative medicines, synthetic biosystems and many others.
Findings and advances in biotechnology do not only pose new possibilities. Challenges arise from the engineering perspective to create biomanufacturing products at large scale and low cost. These two targets are closely related to each other, as the latter constitutes a quintessential criteria for the former. According to \cite{Zhang2017}, the price of a recombinant protein could be reduced from 1,000 \$ pro mg dry protein produced in the laboratory down to 5 \$ pro kg dry protein produced in an industrial scale (see Figure XX). \todo{Recheck the fact, Figure in Appendix} 
Consequently, scientists attempt to scale up cell cultivations, fermentations and other processes. These processes are often carried out in continuous bioreactors, similar to chemical reactors. \cite{Mandenius2016} defined bioreactor as a designed space where biological reactions take place. Its design challenge consists of many aspects, yet most importantly optimizing heat and mass transport between cells and matrix while keeping these living cells intact.