\section{Fluid Dynamic in a Stirred Tank}

\subsection{Description of Turbulence}
The most commonly used method to determine fluid rheology is by means of \textsc{Reynolds}--Number\footnote{Osborne \textsc{Reynolds}, 1842 -- 1912, british physicist, engaged himself mostly on the study and characterization of turbulent flow.}. This dimensionsless number represents the ratio of inertial force to the friction force of the fluid and can be computed as follows:

\begin{align} \label{eq:reynoldGeneral}
    \text{Re} = \frac{w \cdot L_{\text{char}}}{\nu}.
\end{align}

Generally, the velocity of the stirrer blade tip $w_{\text{tip}}$ and stirrer diameter $d$ are used to calculate the \textsc{Reynolds}--Number in a stirred tank. \citet{Kraume2012Transport} and \citet{Wollny2010Diss} stated, at $\text{Re} < 10$ the fluid-flow has its laminar form and becomes entirely turbulent for $\text{Re} > 10^4$. The difference between laminar and turbulent flow are illustrated in Figure \ref{fig:stirredTankLamTurb}. Laminar flows are distinguished in their harmonized movement. The velocity of each volume element forms a streamline throughout the whole vessel. Once Re exceeds its treshold ($\text{Re} > 10$), the turbulent eddy begins to form inside the vessel. This results in a chaotic oscillation of the velocity (see Figure \ref{plot:turbulentFlowVelocity}).

The stirrer \textsc{Reynolds}--Number is defined as:

\begin{gather}
    \text{Re} = \frac{n \cdot d^2}{\nu} = \frac{n \cdot d^2 \cdot \rho}{\eta}.
\end{gather}

Similar to the \textsc{Reynolds}--Number, the power input in a stirred vessel is commonly expressed in its non-dimensional form as the \textsc{Newton}--Number:

\begin{gather}
    \text{Ne} = \frac{P}{\rho \cdot n^3 \cdot d^5} = \frac{2 \cdot \pi \cdot M_t}{\rho \cdot n^2 \cdot d^5}.
\end{gather}

According to \citet{Wollny2010Diss}, this \textsc{Newton}--Number is directly proportional to \textsc{Reynolds}--Number in laminar zone ($\text{Ne} \sim \text{Re}$), whereas such correlation between these two quantities is not to be found in turbulent zone ($\text{Ne} \neq f(\text{Re})$).

\paragraph{The Turbulent Velocity}
In fluid dynamics, turbulent flows are determined by its three-dimensional, localized and instationary eddies. The instantaneous velocity $w_i(\Vec{x},t)$ for a fixed volume element fluctuates around its average $\overline{w}_i(\Vec{x})$ over the course of time due to the turbulent eddy. This actual velocity can then be depicted as the sum of a mean velocity $\overline{w}_i(\Vec{x})$ and a turbulent fluctuation $w'(\Vec{x},t)$ as proposed by Reynold. The mean value is defined as:

\begin{gather}
    \overline{w}_i \equiv \frac{1}{\tau} \cdot \int_0^\tau w(t) \, dt.
\end{gather}

As portrayed in Figure \ref{plot:turbulentFlowVelocity2}, the average turbulent velocity $\overline{w}$ is a function of time for non-stationary process, whereas its value does not change for stationary process. Should the turbulent fluctuation be averaged for a certain period, the result would take a value of zero by definition. Thus, the root-mean-square is used to calculate the average fluctuation instead. Notice that the turbulent velocity takes in three dimensions in space. Provided that the assumption of local isotropy, which \citet{Kolmogorov} theoretically defined for turbulent flows, can be applied, the effective value of turbulent fluctuations is identical for every cartesian coordinates and therefore can be defined as:

\begin{align}
    {w'}_\text{eff} \equiv \sqrt{\overline{(w')^2}} &= \sqrt{\overline{(w')_x^2}} = \sqrt{\overline{(w')_y^2}} = \sqrt{\overline{(w')_z^2}} \nonumber \\
    &= \frac{1}{\tau} \cdot \int_0^\tau \sqrt{\frac{{w'}_\text{x}^2 + {w'}_\text{y}^2 + {w'}_\text{z}^2}{3}} \, dt.
\end{align}

This quantity is necessary in defining the \textit{specific turbulent kinetic energy}:

\begin{align}
        k   &= \frac{{w'}_x^2 + {w'}_y^2 + {w'}_z^2}{2} \nonumber \\
            &= \frac{3}{2} \cdot {w'}^2
\end{align}

The characteristic quantities of turbulent flow, the average velocity $\overline{w}(t)$ and its fluctuations $w'(t)$, are proportional to $w_{\text{tip}}$ for each volume element of the fluid in a stirred tank:
\begin{gather}
    w_\text{tip} = \pi{} \cdot n \cdot d \sim \overline{w} \sim w'.
\end{gather}

\plot{pictures/ch2/sec2_turbulent_flow.tex}{Schematic diagram of turbulent velocity for any volume element of the fluid}{plot:turbulentFlowVelocity}
\plot{pictures/ch2/sec2_turbulent_flow_2.tex}{Schematic diagram of turbulent velocity for any volume element of the fluid for a stationary process (top) and non-stationary process (bottom)}{plot:turbulentFlowVelocity2}

\subsection{Macro and Micro Turbulence}
The rotating movement of the stirrer induce
Biggest turbulent eddies: \citep{Wollny2010Diss}
\begin{align}
    \Lambda \sim \frac{h}{2} = \frac{d}{10}
\end{align}
\vectorGraphic{pictures/ch2/macroAndMicroEddy.pdf}{Size of turbulent eddies}{fig:macroAndMicroEddy}
