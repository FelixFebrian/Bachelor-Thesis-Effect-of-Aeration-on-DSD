\section{Stirred Tank Bioreactor Design}

A bioreactor is generally described as a technical equipment for growing organisms, typically bacteria or yeast, so that its metabolites products or its ability to convert organic waste can be benefited from. The main objectives of bioreactor design can thus derived from this broad description, namely attaining the most ideal conditions for the organism's life sustainability yet maintaining the efficiency of its production yield. Clearly, each bioreactor has to be adapted to its underlying biological system. To achieve these objectives, a bioreactor design process has to address multiple issues, such as the mass and heat transfer, mixing efficiency, scale-up procedure, rheology, and many more. A comprehensive explanation bioreactor design criteria can be found in \citet{Mandenius2016}

At present, a number of design alternatives has emerged to answer these design issues. Some examples include the bubble column reactor, the airlift bioreactor and the wave bioreactor, each design having its own advantages and disadvantages. However, the predominantly used design for submerged culture, with a few exceptions, is the stirred tank bioreactor, which is also the main subject of this work. In spite of its drawback in form of mechanical agitation, the stirred tank is widely chosen due to its versatility, operability and manufacturability. Scale-up processes of a stirred tank has been extensively reviewed as well.

\vectorGraphic{pictures/ch2/stirredTankLamTurb.pdf}{A sketch of a common stirred-tank with its characteristic dimensions (left), schematic illustrations of laminar flow (middle) and turbulent flow (right)}{fig:stirredTankLamTurb}

\paragraph{Technical Stirring Device}
The design of a technical stirring device can be taken from e.g. DIN 28130. Figure \ref{fig:stirredTankLamTurb} depicts its main components including its characteristic dimensions. The tank itself ordinarily holds a cylindrical form either with a curved or a flat bottom. The stirrer is set in the middle of the tank and produces a rotating movement. Under a certain impeller rotational frequency, the fluid movement follows circular streamlines, both at the radial course (primary flux, see Figure \ref{fig:stirredTankLamTurb} middle) and at the axial course (secondary flux, see Figure \ref{fig:stirredTankLamTurb} right). This kind of flow is called \textit{laminar} flow.

Should no further component is build inside the tank, the rotating movement can cause a deformation on the surface of liquid and it creates the so-called \textit{Fl\"ussigkeitstrombe (Liquid Funnel ?)}.