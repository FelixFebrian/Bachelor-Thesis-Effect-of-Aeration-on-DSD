\chapter{Open Questions}

\section{What is the main objective of my BA?}
Using previous works (such as \citet{zhou1998correlation}, \citet{henzler1996modelluntersuchungen}, \citet{langer2000verstandnis}) and their results as foundation, you can conclude that the hydrodynamic-stress caused by stirrer can be modelled using \textsc{Shinnar}'s Model and also can be expressed using the energy dissipation rate $\varepsilon$, so that $d_{32} \propto \varepsilon^a$. This correlation is now studied when a third fluid phase is introduced through aeration.

Shortly, one objective is to find x, where:
\begin{align}
    d_{32} \propto \varepsilon^x
\end{align}
Where the maximum energy dissipation rate is also proportional to the mean energy dissipation rate $\overline{\varepsilon}$, which is also proportional to the gassed power draw $P_\text{gassed}$.

Beyond that, \textit{correlations} or \textit{models} can also be formed which captures the different mechanism on hydromechanical stress on fluid particle.
\vspace{5cm}

\section{What is the essence of the model from \textsc{Shinnar}}
Essentially, the biggest droplet possible in a stirred Fl/Fl-System is to be approximated through the turbulent kinetic energy and the droplet's surface energy.
\begin{align}
    \frac{E_\text{kin}}{E_\sigma} \leqslant 1
\end{align}
\vspace{5cm}

\section{What are \textit{Tr\"agheitsunterbereich} and \textit{Dissipationsbereich}?}
From \citet{Wollny2010Diss}: Der Trägheitsunterbereich zeichnet sich durch turbulent flie\ss{}ende Wirbel und der Dissipationsbereich
durch laminar flie\ss{}enden Wirbel aus -- I cannot imagine a laminary turbulent eddy.
\vspace{5cm}

\section{What is the assumptions from the model from \textsc{Shinnar}}
This model is based on the assumption, that the hydromechanical stress on droplets is solely caused by shear-stress through the movement in continuous phase.
\vspace{5cm}

\section{What are other mechanisms can cause particle stress?}
\begin{itemize}
    \item Expansion flow
    \item Pressure gradient from particle breakups (?)
\end{itemize}
\vspace{5cm}

\section{Particle Stress from Agitator}
\subsection{Mechanical Stress}
\begin{itemize}
    \item Collisions between stirrer and particles
    \item Collisions between particles and other equipments
\end{itemize}
\subsection{Hydromechanical Stress}
\begin{itemize}
    \item Shear-stress through turbulent velocity
    \item others?
\end{itemize}
Problem with direct method -- the ratio of these forces cannot be quantified -- combination with indirect methods (PIV or LIV measurements) would be needed.
\vspace{5cm}

\section{What can I analyse from direct methods?}
From the model from \citet{shinnar1961behaviour}, I can analyse if $d_\text{max} \propto d_{32} \propto \varepsilon^x$ and see if $x$ corresponds the theoritical value. So, from my understanding:
\begin{itemize}
    \item Control: Stirred W/TX-100 + EAL -- Reproduction of Experiments from Wollny, Malik, Biedermann, etc
    \item Stirred W/TX-100 + air -- To see if the Model is also applicable to bubble particles
    \item Stirred W/TX-100 + air + EAL -- To see the effect of bubbly flows to turbulent velocity in addition to the main flow from agitator
    \item (Not sure) W/TX-100 + air + EAL -- To see how much bubbly flows cause turbulent velocity
\end{itemize}
\vspace{5cm}
