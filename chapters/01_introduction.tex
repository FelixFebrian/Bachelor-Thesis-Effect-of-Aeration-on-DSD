\chapter{Introduction} \label{ch:introduction}

\section{Motivation}
The definition of a bioreactor sums up an artificially constructed space where biological reactions are carried out. Hence it should constitute the original ecosystem which provides an ideal environment for the partaking organisms, generally microscopic ones, as thoroughly and sufficiently as possible. This raises several issues regarding life sustainability of the microorganisms in the designed space, for instance mixing efficiency and nutrients supply.

Mixing efficiency can often be intensified through an impeller application as in a stirred-tank bioreactor, which also will be the main subject of this work. However, the stress caused by impeller's stirring-force may hinder the efficiency of the underlying process. Numerous studies contributed to indicate this problem not only in biotechnological, but also in the broader chemical industry. For example, \citet{Frahm2007} and \citet{Henzler1993} addressed animal cell's susceptibility to shear-stress due to the lack of cell walls, that its oxygen supply is limited to bubble-free membrane aeration. Even the more robust plant cell suspension is adversely affected by hydrodynamic-related stress \citep{Eibl2009}. To mention a purely chemical system, \citet{Wollny2010Diss} stated the danger of excessive energy input in the crystallization process to achieve the desired crystal size distribution.

Nutrients demand for aerobic cell cultures includes oxygen requirement conducive to cellular respiration. There are several oxygenation methods to fulfill this demand, the main ones being: 
\begin{inlineenum}
    \item surface aeration,
    \item membrane aeration and
    \item direct bubble aeration.
\end{inlineenum}
While the surface and membrane aeration method produce little mechanical stress, the direct bubble aeration has its advantage in terms of higher gas/liquid-mass transfer and possible oxygen supply rate \citep{Henzler1993}.

An equilibrium of aeration, mixing performance and the therewith induced stress must therefore be attained and has been one of central issue in bioreactor design criteria. While many investigations on impeller induced particle stress have been done, among others by \citet{Wollny2010Diss} and \citet{Malik2018Master}, this work tries to extend the research scope by taking account of bubble aeration. It's objective is to explore the effect of bubble aeration on particle stress in stirred-tank bioreactor.
 

\section{Outline}